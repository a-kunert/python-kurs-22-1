\documentclass[a4paper]{article} 
\usepackage{minted}
\usemintedstyle{friendly}
\addtolength{\hoffset}{-2.25cm}
\addtolength{\textwidth}{4.5cm}
\addtolength{\voffset}{-3.25cm}
\addtolength{\textheight}{5cm}
\setlength{\parskip}{0pt}
\setlength{\parindent}{0in}

%----------------------------------------------------------------------------------------
%	PACKAGES AND OTHER DOCUMENT CONFIGURATIONS
%----------------------------------------------------------------------------------------

\usepackage{blindtext} % Package to generate dummy text
\usepackage{charter} % Use the Charter font
%\usepackage{lmodern} % Use the Charter font
\usepackage[utf8]{inputenc} % Use UTF-8 encoding
\usepackage{microtype} % Slightly tweak font spacing for aesthetics
\usepackage[english, ngerman]{babel} % Language hyphenation and typographical rules
\usepackage{amsthm, amsmath, amssymb} % Mathematical typesetting
\usepackage{float} % Improved interface for floating objects
\usepackage[final, colorlinks = true, 
            linkcolor = black, 
            citecolor = black]{hyperref} % For hyperlinks in the PDF
\usepackage{graphicx, multicol} % Enhanced support for graphics
\usepackage{xcolor} % Driver-independent color extensions
\usepackage{marvosym, wasysym} % More symbols
\usepackage{rotating} % Rotation tools
\usepackage{censor} % Facilities for controlling restricted text
\usepackage{listings, style/lstlisting} % Environment for non-formatted code, !uses style file!
%\usepackage{pseudocode} % Environment for specifying algorithms in a natural way
\usepackage{style/avm} % Environment for f-structures, !uses style file!
\usepackage{booktabs} % Enhances quality of tables
\usepackage{tikz-qtree} % Easy tree drawing tool
\usepackage{ifthen}
\usepackage{lastpage}
\usepackage{titlesec}
\tikzset{every tree node/.style={align=center,anchor=north},
         level distance=2cm} % Configuration for q-trees
\usepackage{style/btree} % Configuration for b-trees and b+-trees, !uses style file!
%\usepackage[backend=biber,style=numeric,
%            sorting=nyt]{biblatex} % Complete reimplementation of bibliographic facilities
%\addbibresource{ecl.bib}
\usepackage{csquotes} % Context sensitive quotation facilities
\usepackage{fancyhdr} % Headers and footers
\pagestyle{fancy} % All pages have headers and footers
\fancyhead{}\renewcommand{\headrulewidth}{0pt} % Blank out the default header
\fancyfoot[L]{\thesheetsubmission{}} % Custom footer text
\fancyfoot[C]{} % Custom footer text
\fancyfoot[R]{\ifthenelse{\pageref{LastPage} > 1}{\footnotesize Seite \thepage{} von \pageref{LastPage} }} % Custom footer text
\newcommand{\note}[1]{\marginpar{\scriptsize \textcolor{red}{#1}}} % Enables comments in red on margin

\titleformat{\section}
{\normalfont\Large\bfseries}{Lösung zu Aufgabe~\thesection}{1em}{\normalsize}
\titlespacing*{\section}{0em}{6ex}{2ex}


\renewcommand{\theenumi}{\alph{enumi}}
\renewcommand\labelenumi{(\theenumi)}

\newcommand*{\sheetdate}[1]{\def\thesheetdate{#1}}
\newcommand*{\sheetsubmission}[1]{\def\thesheetsubmission{#1}}
\newcommand*{\sheetnumber}[1]{\def\thesheetnumber{#1}}


%----------------------------------------------------------------------------------------


\sheetsubmission{}
\sheetnumber{1}
\sheetdate{22.4.2021}


\begin{document}

%-------------------------------
%	TITLE SECTION
%-------------------------------

\fancyhead[C]{}
\hrule \medskip % Upper rule
\begin{minipage}[t]{0.295\textwidth}
\raggedright
\footnotesize
Dr. Aaron Kunert \hfill\\   
aaron.kunert@salemkolleg.de \hfill \\
\end{minipage}
\begin{minipage}[t]{0.4\textwidth} 
\centering 
\large 
Einführung in Python\\ 
\normalsize 
Blatt \thesheetnumber{}\\ 
\end{minipage}
\begin{minipage}[t]{0.295\textwidth} 
\raggedleft
\footnotesize
\thesheetdate{}
\hfill\\
\end{minipage}
\medskip\hrule 
\bigskip

%-------------------------------
%	CONTENTS
%-------------------------------

\section{}
\begin{minted}{python}
first_name = input("Dein Vorname: ")
last_name = input("Dein Nachname: ")
street = input("Deine Straße: ")
house_number = input("Deine Hausnummer: ")
zip_code = input("Deine PLZ: ")
city = input("Deine Stadt: ")

# linefeeds can be created by using the special charakter \n
print(f"{first_name} {last_name}\n{street} {house_number}\n{zip_code} {city}")
\end{minted}

\section{}
\begin{minted}{python}
# Part (a)
length = input("Länge in Meter: ")
length = float(length)
print(f"Die Länge in Kilometern beträgt {length/1000}km")

# Part (b)
temp = input("Temperatur in Grad Celsius: ")
temp = float(temp)
temp = temp + 273.15
temp = int(temp * 100)/100  # my creative way to round to 2 figures
print(f"Die Temperatur in Grad Kelvin beträgt {temp}")

# Part (c)
length = input("Länge in Zentimeter: ")
length = float(length)
length = length/2.54
length = int(length*100)/100  # my creative way to round to 2 figures
print(f"Die Länge in Zoll beträgt {length}")

# Part (d)
temp = input("Temperatur in Grad Celsius: ")
temp = float(temp)
temp = 9/5 * temp + 32  # you find this conversion rule on Wikipedia
temp = int(temp * 100)/100  # my creative way to round to 2 figures
print(f"Die Temperatur in Grad Fahrenheit beträgt {temp}")
\end{minted}

\newpage
\section{} 
\begin{minted}{python}
x = input("Gib die 1. Zahl ein: ")
y = input("Gib die 2. Zahl ein: ")
z = input("Gib die 3. Zahl ein: ")

x = int(x)
y = int(y)
z = int(z)


if x >= y:
    if z >= x:
        result = z
    else:
        result = x
else:
    if z >= y:
        result = z
    else:
        result = y

print(f"Die größte der drei Zahlen ist {result}")
\end{minted}

\section{}
\begin{minted}{python}
counter = 0

for k in range(1, 1000001):
    if k % 100 == 99 and k % 7 == 0:
        counter += 1

print(f"Es gibt {counter} Zahlen, die diese Bedingung erfüllen")
\end{minted}

\end{document}
