\documentclass[a4paper]{article} 
\input{head}

\sheetsubmission{Abgabe bis Di. 23.11.2021 wahlweise per E-Mail oder per Replit-Einladung}
\sheetnumber{2}
\sheetdate{11.11.2021}



\begin{document}

%-------------------------------
%	TITLE SECTION
%-------------------------------

\input{header}

%-------------------------------
%	CONTENTS
%-------------------------------

\section{Einfaches Prognosemodell}
Am 8.11.2021 betrug die 7-Tages-Inzidenz 219 Fälle pro Woche und 100.000 Einwohner. In den 14 Tagen zuvor ist dieser Wert täglich im Schnitt um 4,58\% im Vergleich zum Vortag gewachsen. Unter der Voraussetzung, dass diese Wachstumsrate unverändert bleibt, soll berechnet werden, in wie vielen Tagen ab dem 8.11.2021 die 7-Tages-Inzidenz erstmals den Wert von 
\begin{enumerate}
\item 300
\item 500
\item 1000
\item 10000
\item 100000
\end{enumerate} 
Fälle pro Woche und 100.000 Einwohner überschreitet. Worin liegen die (offensichtlichen) Grenzen dieses Modells?  

\section{Durchschnitt berechnen}
Sei eine Liste von Zahlen gegeben. Berechne den gewichteten Durchschnitt, bei dem jede zweite Zahl in der Liste doppelt gewichtet werden soll.


\section{Liste auf Vorgänger/Nachfolger durchsuchen}
Sei ein Liste von Namen (z.B. \py{["Max", "Lara", "Tom"]}) sowie ein Name von einer deiner Freundinnen gegeben. Schreibe ein Programm, das prüft, ob die Namen von dir und deiner Freundin direkt hintereinander in der Liste vorkommen. 


\section{Das Collatz-Problem}
Sei $n$ eine beliebige, positive ganze Zahl. Folgende Vorschrift wird auf $n$ angewendet: Ist $n$ gerade, so ersetze man $n$ durch $\frac{n}{2}$, ist $n$ ungerade, so durch $3n + 1$. Setzt man diesen Prozess immer weiter fort, so erhält man eine Folge von positiven ganzen Zahlen. Sobald man bei der Zahl $1$ ankommt, wird dieser Prozess abgebrochen und die Folge gilt als beendet. Die sogenannte \emph{Collatz-Vermutung} besagt, dass diese Folge für jedes $n$ am Ende bei der Zahl $1$ ankommt. 

\vspace{2pt}

Zeige, dass die Vermutung für den speziellen Fall $n = 1.000.000.000$ korrekt ist. Wie lange ist diese Folge in diesem Fall, bis sie bei $1$ endet?


\vspace{2pt}

{\footnotesize\textbf{Beispiel:}
 Startet man bei $n = 20$ erhält man die Folge: $20, 10, 5, 16, 8, 4, 2, 1$. Die Folge hat hier die Länge $8$.}  

\vspace{2pt}

{\footnotesize\textbf{Bemerkung:}
Die Collatz-Vermutung ist bis heute ungelöst und inzwischen sind über 1.000.000 Euro Preisgeld für eine Lösung ausgehoben.}  
 


  








\end{document}
