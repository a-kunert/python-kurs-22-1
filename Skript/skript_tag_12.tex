\section{Projekt: Geografie-Quiz}


\begin{frame}
\begin{block}{Das \emph{minimal viable product} (MVP)}
\vspace{2pt}
Anforderungen: 
\pause 
\begin{itemize}[<+->]
	\item Es werden Länder aus einem JSON gelesen.
	\item Zu einem Land aus der Liste wird die Hauptstadt abgefragt.
	\item Es erscheinen 4 Lösungsmöglichkeiten (Multiple Choice).
	\item Durch Eingabe einer Zahl zwischen 1 bis 4 kann getippt werden. 
	\item Nach Eingabe erscheint ein kurzes Feedback (Richtig/Falsch).  
\end{itemize}
\end{block}
\end{frame}
	
\begin{fragile}
	
\begin{exampleblock}{Beispielausgabe}
\vspace{8pt}

\begin{overprint}
\onslide<1|handout:1>

\console{Was ist die Hauptstadt von Frankreich?}

\console{(1) Bratislava}\\
\console{(2) Berlin} \\
\console{(3) Paris} \\
\console{(4) Stockholm} \\ \\
 
\console{Antwort:} 
\onslide<2|handout:2>

\console{Was ist die Hauptstadt von Frankreich?}

\console{(1) Bratislava}\\
\console{(2) Berlin} \\
\console{(3) Paris} \\
\console{(4) Stockholm} \\ \\
 
\console{Antwort: 3} 
\onslide<3|handout:3>

\console{Was ist die Hauptstadt von Frankreich?}

\console{(1) Bratislava}\\
\console{(2) Berlin} \\
\console{(3) Paris} \\
\console{(4) Stockholm} \\ \\
 
\console{Antwort: 3} \\ \\

\console{Das war korrekt!}
\end{overprint}
\end{exampleblock}
	
\end{fragile}

%\begin{fragile}
%
%\begin{block}{Stand vom letzten Mal}
%\begin{minted}[linenos]{python}
%import json
%import random
%
%# Einlesen der Länderliste
%with open("countries.json") as my_file:
%  countries = json.load(my_file)
%country = countries[0]
%
%# Liste von Hauptstädten
%capitals = []
%for current_country in countries:
%  capitals.append(current_country["capital"])
%
%
%random.shuffle(capitals)
%answer_options = capitals[0:3]
%answer_options.append(country["capital"])
%random.shuffle(answer_options)
%print("\n\n")
%print(f"Was ist die Hauptstadt von {country['name']}?\n")
%for index, option in enumerate(answer_options):
%  print(f"({index + 1}) {option}")
%\end{minted}
%\end{block}
	
%\end{fragile}

\begin{frame}{Übung}

\begin{block}{Multiple Choice Frage}
\vspace{2pt}
Lies die Datei \py{"countries.json" } ein. Nimm das erste Land aus der Liste (Afghanistan) und stelle die Frage nach der Hauptstadt mit 4 Antwortmöglichkeiten wie oben auf der Konsole dar. 

\small{\textbf{Tip}: Aufgabe 4 vom letzten Übungsblatt kann helfen}
\end{block}
\end{frame}


\begin{frame}<beamer:0>[fragile]{Lösung}
\begin{solutionblock}{Multiple Choice Frage}
\begin{minted}{python}
import random
import json
with open("countries.json") as file: 
    countries = json.load(file)

capitals = []
for country in countries: 
    capitals.append(country["capital"])

capital = capitals[0]
random.shuffle(capitals)
answer_options = capitals[0:3]
answer_options.append(capital)
random.shuffle(answer_options)
country = countries[0]

print(f"Was ist die Hauptstadt von { country['name'] }?\n")
for index,option in enumerate(answer_options): 
    print(f"({ index + 1 }) { option }")

\end{minted}
\end{solutionblock}
\end{frame}



\begin{frame}
\begin{block}{Welche Verbesserungen sind dringend nötig?}
	\vspace{2pt}
\pause 
Schritte nach Priorität: 
\pause 
	\begin{enumerate}[<+->]
		\item Antwort einlesen und Feedback geben (Antwort war richtig/falsch)
		\item Statt Afghanistan soll ein zufälliges Land abgefragt werden
		\item Sicherstellen, dass die Antwort-Optionen stets paarweise unterschiedlich sind
	\end{enumerate}	
\end{block}
\end{frame}


\begin{frame}{Übungen}
	
	\begin{block}{Antwort einlesen}
		\vspace{2pt}
		Frage eine Antwort ab. 	
	\end{block}
\vspace{12pt}
	
	\begin{block}{Antwort prüfen}
		\vspace{2pt}
		Entscheide, ob die Antwort richtig oder falsch ist, und gib das Ergebnis auf der Konsole aus. 
	\end{block}
\vspace{12pt}
	
	\begin{block}{Land zufällig auswählen}
		\vspace{2pt}
		Sorge dafür, dass das abgefragte Land zufällig ausgewählt wird. 
	\end{block}
	
	
\end{frame}

\begin{frame}<beamer:0>[fragile]{Lösungen}
	
\begin{solutionblock}{Antwort einlesen}
\begin{minted}{python}
...
for index, option in enumerate(answer_options):
  print(f"({index + 1}) {option}")
answer = input("Antwort: ")
\end{minted}
\end{solutionblock}

\vspace{12pt}

\begin{solutionblock}{Antwort prüfen}
\begin{minted}{python}
...
answer = input("\nAntwort: ")
index = int(answer) - 1
if answer_options[index] == country["capital"]:
  print("Das war korrekt.")
else:
  print("Das war leider falsch.")
\end{minted}
\end{solutionblock}


\end{frame}

\begin{frame}<beamer:0>[fragile]{Lösungen}
	
\begin{solutionblock}{Land zufällig auswählen}
\begin{minted}{python}
...
with open("countries.json") as my_file:
  countries = json.load(my_file)
random.shuffle(countries)
country = countries[0]
...
\end{minted}
\end{solutionblock}
\end{frame}


\begin{frame}
\begin{alertblock}{Möglicher Bug}
\vspace{2pt}
Um die Lösungsmöglichkeiten zu generieren, werden 3 Haupstädte zufällig ausgewählt und dann die korrekte Hauptstadt hinzugefügt. 
Theoeretisch ist es möglich, dass die korrekte Hauptstadt schon bei den 3 zufälligen Haupstädten dabei war. 
\end{alertblock}
\pause
\vspace{12pt} 
\begin{exampleblock}{Ist nur ein kleines Problem}
\vspace{2pt}
Das ist nur ein kosmetisches Problem. Die Funktionalität geht davon nicht kaputt. 
\end{exampleblock}
\end{frame}


\begin{fragile}
\begin{block}{Mögliche Lösung}
\vspace{2pt}
Mittels \pybw{.remove()} kann man die korrekte Hauptstadt aus der Liste \pybw{capitals} entfernen. 
\end{block}

\pause 
\vspace{12pt}
\begin{block}{Bessere Lösung}
\vspace{2pt}
\begin{minted}{python}
...
for current_country in countries:
  if current_country["capital"] != country["capital"]:
    capitals.append(current_country["capital"])
...
\end{minted}
\end{block}

\pause 
\vspace{12pt}


Warum ist die zweite Lösung besser?


\end{fragile}

\begin{frame}{Übung}

\begin{block}{Feature Request}
\vspace{2pt}
Es sollen (zunächst) nur Hauptstädte aus Europa abgefragt werden.
\end{block}
	
\end{frame}



\begin{frame}<beamer:0>[fragile]{Lösung}
	
\begin{solutionblock}{Feature Request: Nur europäische Hauptstädte}
\begin{minted}{python}
...
with open("countries.json") as my_file:
  countries = json.load(my_file)

filtered_countries = []
for country in countries: 
  if country["continent"] == "Europa": 
    filtered_countries.append(country)

countries = filtered_countries

random.shuffle(countries)
...
\end{minted}
\end{solutionblock}
\end{frame}

\section{Refactoring \\ \footnotesize{Wie man immer mehr Feature Requests unterbringt}}

\begin{frame}
	
\begin{block}{Mehrere Fragen}
\vspace{2pt}
Ein Quiz mit nur einer Frage ist schon sehr lame. Daher sollen ab jetzt nacheinander 10 Fragen gestellt werden. 	
\end{block}

\pause 
\vspace{12pt}

\begin{alertblock}{Problem}
\vspace{2pt}
Das ganze wird langsam unübersichtlich.	
\end{alertblock}

\pause
\vspace{12pt}
\begin{exampleblock}{Lösung}
	\vspace{2pt}
Refactoring
\end{exampleblock}

\end{frame}

\begin{frame}
\metroset{block=fill}
\begin{block}{Definition}
\vspace{2pt}
Als \emph{Refactoring} von Code wird das Aufräumen bzw. Umstrukturieren von Code genannt, um ihn übersichtlicher und leichter erweiterbar zu machen. 
Die Funktionalität des Codes wird dabei nicht verändert. 
\pause

Typische Maßnahmen sind: 
\pause 
\begin{itemize}[<+->]
	\item Hinzufügen von Kommentaren
	\item Struktur durch Leerzeilen
	\item Umbennung von Variablen/Funktionen
	\item Zerlegung des Codes in kleinere Schritte
	\item Extraktion von gleichen Werten in eine Variable
	\item Extraktion von Arbeitsschritten in Funktionen
\end{itemize}
\end{block}
\end{frame}



\begin{frame}
	
\begin{block}{Arbeitsschritte in unserem Projekt}
\vspace{2pt}
\pause 
\begin{enumerate}[<+->]
	\item Einlesen aller Länder
	\item Länder filtern (nur Europa)
	\item Ein Land (\pybw{country}) zufällig wählen
	\item Liste der Lösungsmöglichkeiten anlegen (in Abhängigkeit von \pybw{country})
	\item Frage stellen
	\item Antwort auswerten/Feedback geben
\end{enumerate}
\end{block}
	
\end{frame}

\begin{frame}
	\begin{block}{Mögliche Maßnahmen}
		\vspace{2pt}
	\pause 
	\begin{itemize}[<+->]
	\item Umbenennung von \pybw{country} in \pybw{active_country}.
	\item Extraktion des Einlesens und Filterns in Funktionen.
    \item Extraktion der Erstellung der Lösungsmöglichkeiten in eine Funktion
    \item Extraktion der Anzeige der Frage in eine Funktion
    \item Extraktion der Auswertung der Antwort in eine Funktion 
	\end{itemize}	
	\end{block}
\end{frame}


\begin{fragile}

\begin{block}{Einlesen}
\vspace{2pt}
\begin{minted}{python}
def get_countries():
  with open("countries.json") as my_file:
    return json.load(my_file)
\end{minted}
\end{block}

\vspace{12pt}
\pause 

\begin{block}{Filtern}
\vspace{2pt}
\begin{minted}{python}
def filter_countries(countries, continent):
  result = []
  for country in countries:
    if country["continent"] == continent:
      result.append(country)
  return result
\end{minted}
\end{block}

\end{fragile}

\begin{fragile}
	
\begin{block}{Erstellung der Antwort-Optionen}
\vspace{2pt}
\begin{minted}{python}
def get_answer_options(active_country, countries):
  capitals = []
  for country in countries:
    if country["capital"] != active_country["capital"]:
      capitals.append(country["capital"])
  random.shuffle(capitals)
  answer_options = capitals[0:3]
  answer_options.append(active_country["capital"])
  random.shuffle(answer_options)
  return answer_options
\end{minted}
\end{block}

\end{fragile}

\begin{fragile}
\begin{block}{Anzeige der Frage}
\vspace{2pt}
\begin{minted}{python}
def display_question(active_country, answer_options):
  print("\n\n")
  print(f"Was ist die Hauptstadt von {active_country['name']}?\n")
  for index, option in enumerate(answer_options):
    print(f"({index + 1}) {option}")
\end{minted}
\end{block}

\vspace{12pt}
\pause 

\begin{block}{Auswertung der Antwort}
	\vspace{2pt}
	\begin{minted}{python}
def check_answer(active_country, answer_options):
  answer = input("\nAntwort: ")
  index = int(answer) - 1
  if answer_options[index] == active_country["capital"]:
    print("Das war korrekt.")
  else:
    print("Das war leider falsch.")
\end{minted}
\end{block}

\end{fragile}


\begin{fragile}
\begin{block}{Programmablauf}
\vspace{2pt}
\begin{minted}{python}
...	
def get_countries():
...
def filter_countries(countries, continent):
...
def get_answer_options(active_country, countries):
...
def display_question(active_country, answer_options):
...
def check_answer(active_country, answer_options):
...
countries = get_countries()
countries = filter_countries(countries, "Europa")
random.shuffle(countries)
active_country = countries[0]
answer_options = get_answer_options(active_country, countries)
display_question(active_country, answer_options)
check_answer(active_country, answer_options)	
\end{minted}
\end{block}
\end{fragile}


\begin{frame}{Übung}
\begin{block}{Mehrere Fragen}
\vspace{2pt}
Schreibe das Programm so um, dass nicht nur eine Frage, sondern 10 (verschiedene) Fragen gestellt werden. 
\end{block}
\end{frame}



\begin{frame}<beamer:0>[fragile]{Lösung}
	
\begin{solutionblock}{}
\begin{minted}{python}
...
random.shuffle(countries)
for index in range(0,10):
  active_country = countries[index]
  answer_options = get_answer_options(active_country, countries)
  display_question(active_country, answer_options)
  check_answer(active_country, answer_options)
\end{minted}
\end{solutionblock}
\end{frame}


\begin{fragile}
\begin{block}{Ein bisschen Kosmetik}
\vspace{2pt}
Je nach Größe der Konsole sieht man noch Teile der alten Frage, oder aber man sieht das Feedback nicht mehr richtig. 
Dies lässt sich durch eine schlaue Verteilung von Leerzeilen und einem Bestätigungsdialog lösen: 

\begin{minted}{python}
...
  print("\n"*100)
  print(f"Was ist die Hauptstadt von {country['name']}?\n")
...
...
  score = score + check_answer(active_country, answer_options)
  input("\nWeiter mit beliebiger Taste")
\end{minted}

\end{block}
\end{fragile}


\begin{frame}{Übung}
\begin{block}{Feature Request: Korrektur}
\vspace{2pt}
Statt nur dem Hinweis \texttt{"Das war leider falsch"} soll nun auch noch zusätzlich die richtige Antwort ausgegeben werden. 
\vspace{2pt}

Beispiel: 

\texttt{Das war leider falsch. Die richtige Antwort wäre "{}Athen"{} gewesen.} 
\end{block}
\end{frame}


\begin{frame}<beamer:0>[fragile]{Lösung}

\begin{solutionblock}{}
\begin{minted}{python}
...
def check_answer(active_country, answer_options):
  answer = input("\nAntwort: ")
  index = int(answer) - 1
  correct_answer = active_country["capital"]
  if answer_options[index] == correct_answer:
    print("Das war korrekt.")
    return 1
  else:
    print(f"Das war leider falsch. Die richtige Antwort wäre \"{correct_answer}\" gewesen.")
    return 0
...
\end{minted}
\end{solutionblock}
\end{frame}

\begin{frame}{Übung}
\begin{block}{Feature Request: Punktzahl}
\vspace{2pt}
Am Ende des Spiels soll angezeigt werden, wie viele Antworten richtig waren. 
\end{block}

\pause 
\vspace{12pt}
\begin{exampleblock}{Hinweis}
\vspace{2pt}
Führe eine globale Variable \pybw{score} ein und verwende entsprechende Rückgabewerte in der Funktion \pybw{check_answer}.	
\end{exampleblock}



\end{frame}

\begin{frame}<beamer:0>[fragile]{Lösung}
	
\begin{solutionblock}{}
\begin{minted}{python}
score = 0
...
def check_answer(active_country, answer_options):
  answer = input("\nAntwort: ")
  index = int(answer) - 1
  if answer_options[index] == active_country["capital"]:
    print("Das war korrekt.")
    return 1
  else:
    print("Das war leider falsch.")
    return 0
...
for index in range(0,10): 
  ...
  score = score + check_answer(active_country, answer_options)
  ...
print(f"\n\nDu hast {score}/10 Fragen richtig")
\end{minted}
\end{solutionblock}
\end{frame}






\begin{frame}{Übung}
\begin{block}{Feature Request: Mehr Kontinente}
\vspace{2pt}
Wie muss man den Code umschreiben, damit Länder aus Europa \emph{und} Asien abgefragt werden.
\end{block}


\pause

\vspace{12pt}

\begin{exampleblock}{Hinweis}
	\vspace{2pt}
Schreibe die Funktion \pybw{filter_countries} so um, dass sie statt einem String \pybw{continent} eine Liste \pybw{continents} erwartet. 
\end{exampleblock}

\end{frame}

\begin{frame}<beamer:0>[fragile]{Lösung}
	
\begin{solutionblock}{}
\begin{minted}{python}
...
def filter_countries(countries, continents):
  result = []
  for country in countries:
    if country["continent"] in continents:
      result.append(country)
  return result
...
countries = get_countries()
countries = filter_countries(countries, ["Europa", "Asien"])
...
\end{minted}
\end{solutionblock}
\end{frame}


\begin{frame}
\begin{block}{Weitere Verbesserungsideen}
	\pause 
\begin{itemize}[<+->]
  \item Besseres Fehlerhandling (was ist, wenn die Antwort nicht zwischen 1 und 4 ist?)
  \item Anzahl der Fragen variabel machen. 
  \item Schwierigkeitsgrad variabel (d.h. Anzahl der Antwortmöglichkeiten) variabel machen
  \item Spiel vorzeitig beendbar machen. 
  \item Menü zu Beginn, wo man die Kontinente angeben kann (Mehrfachauswahl)
  \item Neuer Fragentyp: Anzahl der Einwohner. Hier kann man das Mutliple-Choice anders aufbauen, oder durch explizite Antworten ersetzen. 
  \item Falsch beantwortete Fragen in einer Trainingsrunde wiederholen.  
  \item Den Punktestand unter einem Username abspeichern.
  \item Punktzahl von der Reaktionszeit abhängig machen. 
\end{itemize}
\end{block}
\end{frame}