\section{Den Schleifenfluss kontrollieren  \\ \footnotesize \texttt{break}, \texttt{continue} und \texttt{else}}


\begin{fragile}
	
\metroset{block=fill}

\begin{block}{Das \texttt{break}-Statement}
Taucht innerhalb einer Schleife das Schlüsselwort \py{break} auf, so wird die weitere Abarbeitung der Schleife abgebrochen. Die Ausführung wird mit dem Code \emph{nach} dem Schleifenblock ausgeführt. 		
\end{block}

\metroset{block=transparent}

\vspace{12pt} \pause 


\begin{exampleblock}{Beispiel}
\vspace{2pt}

\begin{overprint}
	\onslide<2|handout:0>
\begin{minted}{python}
for k in range(1,100):
  print(k)
  if k > 3:
    break
print("fertig")
\end{minted}
\onslide<3|handout:1>
\begin{minted}{python}
for k in range(1,100):
  print(k)
  if k > 3:
    break
print("fertig")
# 1 2 3 4 
# fertig
\end{minted}
\end{overprint}

\end{exampleblock}

	
\end{fragile}



\begin{fragile}
	
\metroset{block=fill}

\begin{block}{Das \texttt{continue}-Statement}
Taucht innerhalb einer Schleife das Schlüsselwort \py{continue} auf, so wird der aktuelle Schleifendurchgang abgebrochen. Die Ausführung wird mit der nächsten Schleifeniteration fortgesetzt. 
\end{block}

\metroset{block=transparent}

\vspace{12pt} \pause 


\begin{exampleblock}{Beispiel}
\vspace{2pt}

\begin{overprint}
\onslide<2|handout:0>
\begin{minted}{python}
for k in range(1,11):
  if k % 2 == 0:
    continue
  print(k)
\end{minted}
\onslide<3|handout:1>
\begin{minted}{python}
for k in range(1,11):
  if k % 2 == 0:
    continue
  print(k)
# 1 3 5 7 9 
\end{minted}
\end{overprint}
\end{exampleblock}
	
	
\end{fragile}


\begin{fragile}
	
\metroset{block=fill}

\begin{block}{Der \texttt{else}-Block einer Schleife}
Analog zum \py{if}-Statement, kann auch eine Schleife einen \py{else}-Block haben. Dieser wird ausgeführt, wenn die Schleife \emph{regulär} (also nicht durch die Verwendung von \py{break}) beendet wird.  
\end{block}

\metroset{block=transparent}

\vspace{12pt} \pause 


\begin{exampleblock}{Beispiel}
\begin{minted}{python}
name = input("Dein Name: ")

for letter in name: 
  if letter == "a" or letter == "A":
    print("Dein Name enthält ein A")
    break
else: 
  print("Dein Name enthält kein A")
\end{minted}
\end{exampleblock}
	
	
\end{fragile}



\begin{fragile}[Übungen]

\begin{block}{Zählen bis zur nächsten 10er-Zahl}
	\vspace{2pt}
Lies eine Zahl \pybw{x} ein und gib auf der Konsole die Zahlen von \pybw{x} bis zur nächsten 10er-Zahl aus. 
\\
Ist die Eingabe \pybw{x = 17}, so soll die Ausgabe wie folgt aussehen: 

\console{17}\\
\console{18}\\
\console{19}\\
\console{20}
\end{block}
	
\vspace{12pt}
\pause 

\begin{block}{Zählen mit Lücken}
	\vspace{2pt}
	Schreibe ein Skript, dass die Zahlen von 1 bis 99 aufzählt, dabei allerdings die 10er-Zahlen weglässt. Verwende dabei ein \pybw{continue}-Statement.
\end{block}
\end{fragile}

\begin{frame}<beamer:0>[fragile]{Lösungen}

\begin{solutionblock}{Zählen bis zur nächsten 10er-Zahl}
\begin{minted}{python}
x = input("Gib eine Zahl an: ")
x = int(x)

for k in range(x, x + 11):
  print(k)
  if k % 10 == 0:
    break
\end{minted}
\end{solutionblock}

\vspace{12pt}

\begin{solutionblock}{Zählen mit Lücken}
\begin{minted}{python}
for k in range(1, 100):
  if k % 10 == 0:
    continue
  print(k)
\end{minted}
\end{solutionblock}

\end{frame}

\begin{fragile}[Übung]
\begin{block}{Quizfrage mit Ausstiegsmöglichkeit}
\vspace{2pt}
Schreibe ein Programm, dass solange nach einer Hauptstadt Deiner Wahl fragt, bis die richtige Antwort eingegeben wird. Wird allerdings der Buchstabe \pybw{q} eingegeben, so bricht das Programm ab. 

\vspace{12pt}

\begin{solutionblock}{Quizfrage mit Ausstiegsmöglichkeit}
\begin{minted}{python}
answer = input("Was ist die Hauptstadt von Frankreich?")
while answer != "Paris": 
  print("Das war leider falsch, versuch es gleich nochmal")
  answer = input("")
  if answer == "q": 
    break
else: 
  print("Das war richtig!")
\end{minted}
\end{solutionblock}
\end{block}
\end{fragile}




\begin{fragile}[Harte Übung]
\begin{block}{Primzahltest}
\vspace{2pt}
Lies eine ganze Zahl \py{x} ein und überprüfe, ob diese Zahl eine Primzahl ist. Die Ausgabe des Programms soll etwa wie folgt aussehen:  

\console{Die Zahl 28061983 ist eine Primzahl.}
\end{block}

\vspace{12pt}
\begin{solutionblock}{Lösung}
\begin{minted}{python}
x = input("Gib eine Zahl ein: ")
x = int(x)

for k in range(2, x):
  if x % k == 0:
    print(f"{x} ist keine Primzahl.")
    break
else:
  print(f"{x} ist eine Primzahl.")
\end{minted}
\end{solutionblock}

\end{fragile}



\section{Listen \\ \footnotesize Viele Variablen gleichzeitig speichern}


\begin{frame}
\begin{block}{Problemstellung}
\vspace{2pt}
Lies mit Hilfe einer Schleife nach und nach Ländernamen ein. 
Alle Länder sollen dabei gespeichert werden. Danach sollst Du die Möglichkeit haben, das soundsovielte Land anzeigen lassen zu können.   

\vspace{8pt}

Wie macht man das? 
\end{block}
\end{frame}

\begin{fragile}{}
\begin{block}{Lösung \footnotesize(fast)}
\begin{minted}{python}
# ...
# Um das Eingaben der Länder kümmern wir uns noch 
countries = ["Deutschland", "Frankreich", "Italien", "Spanien"] 

index = input("Das wievielte Land möchtest Du nocheinmal anschauen?")
index = int(index)

print(f"Das { index }. Land ist { countries[index] }")
\end{minted}
\end{block}
\end{fragile}


\begin{fragile}

\metroset{block=fill}
\begin{block}{Struktur einer \emph{Liste}}
\vspace{2pt}
\large
\texttt{my\_list = }\pause {\Large\texttt{[}}\pause 
\texttt{element\_0}\pause,
\pause 
\texttt{element\_1}, \pause 
 \dots   
, \texttt{element\_n}\pause \Large{\texttt{]}}
\end{block}

\pause 

Die Variable \py{my_list} trägt nicht nur einen Wert, sondern $n+1$ Werte. Ansonsten verhält sich \py{my_list} wie eine ganz \enquote{normale} Variable. 
Als Einträge einer Liste sind beliebige Werte mit beliebigen Datentypen zugelassen. 


\vspace{12pt}

\pause

\textbf{Frage:} Welchen Datentyp hat die Liste \py{[2, 2.3, "Hello"]} ? 
	
\end{fragile}

\begin{frame}
	
\begin{block}{Auf Listenelemente zugreifen}
	
\vspace{2pt}

Auf das \pybw{n}-te Element der Liste \py{my_list} kann man mittels \py{my_list[n]} zugreifen. 

\pause 

Mit \py{my_list[-1]}, \py{my_list[-2]}, etc. kann man auf das letzte, vorletzte, etc. Element 
der Liste zugreifen. 

\end{block}

\pause 
\vspace{12pt}

\begin{alertblock}{Achtung}
\vspace{2pt}
Python fängt bei 0 an zu zählen. D.h. das erste Element in der Liste hat den Index 0. \\
Beispiel: \py{my_list[1]} liefert das \textbf{2. Element} der Liste. 
\end{alertblock}

	
\end{frame}	


\begin{frame}
\begin{block}{Schreibzugriff auf Listenelemente}
\vspace{2pt}
Nach dem gleichen Prinzip lassen sich einzelne Listeneinträge verändern. \\
Beispiel: \py{my_list[3] = "Albanien"}. 
\end{block}

\pause 
\vspace{12pt}



\begin{alertblock}{Achtung}
\vspace{2pt}
Man kann nur schon existierende Listeneinträge verändern. 
\end{alertblock}

\pause 
\vspace{12pt}

\end{frame}


\begin{frame}
\begin{block}{Listeneinträge hinzufügen}
	\vspace{2pt}
	Mit der \emph{Methode} \pybw{.append()} kann ein Eintrag zur Liste hinzugefügt werden. \\ 
	Bsp: \py{my_list.append("Russland")} fügt den String \py{"Russland"} zu der Liste hinzu. 
\end{block}	

\pause 
\vspace{12pt}


\begin{block}{Listeneinträge entfernen}
	\pause 
\vspace{2pt}
Mit dem Keyword \pybw{del} kann man Einträge an einer bestimmten Position löschen. Dabei verschieben sich die darauffolgenden Einträge um \pybw{1} nach vorne. \\
Beispiel: \py{del my_list[2]} löscht das dritte Element.  

\pause 

Mit der Methode \pybw{.remove()} kann man Einträge mit einem bestimmten Wert löschen. \\
Beispiel: \py{my_list.remove("Italien")} entfernt den ersten Eintrag mit dem Wert \py{"Italien"}. Ist der Wert nicht vorhanden gibt es eine Fehlermeldung. 
\end{block}
\end{frame}

\begin{fragile}[Übung]
\begin{block}{Eine Liste erstellen}
\vspace{2pt}
Schreibe ein kleines Programm, dass Dich ca. 4x nach einem Land fragt, das Du besucht hast und Dir am Ende die Liste der besuchten Länder ausgibt. 	
\end{block}
\vspace{12pt}
\begin{solutionblock}{Lösung}
\begin{minted}{python}
countries = []
for k in range(1, 5):
  country = input("Wo warst Du schonmal im Urlaub? ")
  countries.append(country)
print(countries)
\end{minted}
\end{solutionblock}
\end{fragile}





\begin{fragile}[Übung]
\begin{block}{Das Eingangsproblem}
\vspace{2pt}
Schreibe ein kleines Programm, dass solange Namen von Ländern einliest, bis Du \textbf{q} drückst. Danach sollst Du die Möglichkeit haben, eine Zahl \pybw{k} einzugeben, so dass das  \pybw{k}-te Land angezeigt wird. 
\end{block}	
\end{fragile}

\begin{frame}<beamer:0>[fragile]{Lösung}
\begin{solutionblock}{Das Eingangsproblem}
\begin{minted}{python}
countries = []
while True:
  country = input("Gib ein Land ein: ")
  if country == "q":
    break
  countries.append(country)

index = input("Das wievielte Land möchtest Du nochmal anschauen?")
index = int(index)
print(f"Das { index }. Land ist { countries[index-1] }.")
\end{minted}
\end{solutionblock}
\end{frame}







